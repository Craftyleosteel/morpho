\hypertarget{errors}{%
\section{Errors}\label{errors}}

When an error occurs in running a morpho program, an error message is
displayed together with an explanation of where in the program that the
error happened.

\hypertarget{alloc}{%
\subsection{Alloc}\label{alloc}}

This error may occur when creating new objects or resizing them. It
typically indicates that the computer is under memory pressure.

\hypertarget{intrnl}{%
\subsection{Intrnl}\label{intrnl}}

This error indicates an internal problem with morpho. Please contact the
developers for support.

\hypertarget{invldop}{%
\subsection{InvldOp}\label{invldop}}

This error occurs when an operator like \texttt{+} or \texttt{-} is
given operands that it doesn't understand. For example,

\begin{lstlisting}
print "Hello" * "Goodbye" // Causes 'InvldOp'
\end{lstlisting}

causes this error because the multiplication operator doesn't know how
to multiply strings.

If the operands are objects, this means that the objects don't provide a
method for the requested operation, e.g.~for

\begin{lstlisting}
print object1 / object2
\end{lstlisting}

\texttt{object1} would need to provide a \texttt{div()} method that can
successfully handle \texttt{object2}.

\hypertarget{cnctfld}{%
\subsection{CnctFld}\label{cnctfld}}

This error occurs when concatenation of strings or other objects fails,
typically because of low memory.

\hypertarget{uncallable}{%
\subsection{Uncallable}\label{uncallable}}

This error occurs when you try to call something that isn't a method or
a function. Here, we initialize a variable with a string and call it:

\begin{lstlisting}
var f = "Not a function"
f() // Causes 'Uncallable'
\end{lstlisting}

\hypertarget{glblrtrn}{%
\subsection{GlblRtrn}\label{glblrtrn}}

This error occurs when morpho encounters a \texttt{return} keyword
outside of a function or method definition.

\hypertarget{instfail}{%
\subsection{InstFail}\label{instfail}}

This error occurs when morpho tried to create a new object, but
something went wrong.

\hypertarget{notanobj}{%
\subsection{NotAnObj}\label{notanobj}}

This error occurs if you try to access a property of something that
isn't an object:

\begin{lstlisting}
var a = 1
a.size = 5
\end{lstlisting}

\hypertarget{objlcksprp}{%
\subsection{ObjLcksPrp}\label{objlcksprp}}

This error occurs if you try to access a property or method that hasn't
been defined for an object:

\begin{lstlisting}
var a = Object()
print a.pifflepaffle
\end{lstlisting}

or

\begin{lstlisting}
print a.foo()
\end{lstlisting}

\hypertarget{noinit}{%
\subsection{NoInit}\label{noinit}}

This error can occur if you try to create a new object from a class that
doesn't have an \texttt{init} method:

\begin{lstlisting}
class Foo { }
var a = Foo(0.3)
\end{lstlisting}

Here, the argument to \texttt{Foo} causes the \texttt{NoInit} error
because no \texttt{init} method is available to process it.

\hypertarget{notaninst}{%
\subsection{NotAnInst}\label{notaninst}}

This error occurs if you try to invoke a method on something that isn't
an object:

\begin{lstlisting}
var a = 4
print a.foo()
\end{lstlisting}

\hypertarget{clsslcksmthd}{%
\subsection{ClssLcksMthd}\label{clsslcksmthd}}

This error occurs if you try to invoke a method on a class that doesn't
exist:

\begin{lstlisting}
class Foo { }
print Foo.foo()
\end{lstlisting}

\hypertarget{invldargs}{%
\subsection{InvldArgs}\label{invldargs}}

This error occurs if you call a function with the wrong number of
arguments:

\begin{lstlisting}
fn f(x) { return x }
f(1,2)
\end{lstlisting}

\hypertarget{notindxbl}{%
\subsection{NotIndxbl}\label{notindxbl}}

This error occurs if you try to index something that isn't a collection:

\begin{lstlisting}
var a = 0.3
print a[1]
\end{lstlisting}

\hypertarget{indxbnds}{%
\subsection{IndxBnds}\label{indxbnds}}

This error can occur when selecting an entry from a collection object
(such as a list) if the index supplied is bigger than the number of
entries:

\begin{lstlisting}
var a = [1,2,3]
print a[10]
\end{lstlisting}

\hypertarget{nonnmindx}{%
\subsection{NonNmIndx}\label{nonnmindx}}

This error occurs if you try to index an array with a non-numerical
index:

\begin{lstlisting}
var a[2,2]
print a["foo","bar"]
\end{lstlisting}

\hypertarget{arraydim}{%
\subsection{ArrayDim}\label{arraydim}}

This error occurs if you try to index an array with the wrong number of
indices:

\begin{lstlisting}
var a[2,2]
print a[1]
\end{lstlisting}

\hypertarget{dbgquit}{%
\subsection{DbgQuit}\label{dbgquit}}

This notification is generated after selecting \texttt{Quit} within the
debugger. Execution of the program is halted and control returns to the
user.

\hypertarget{symblundf}{%
\subsection{SymblUndf}\label{symblundf}}

This error occurs if you refer to something that has not been previously
declared, for example trying to use a variable of call a function that
doesn't exist. It's possible that the symbol is spelt incorrectly, or
that the capitalization doesn't match the definition (\emph{morpho}
symbols are case-sensitive).

A common problem is to try to assign to a variable that hasn't yet been
declared:

\begin{lstlisting}
a = 5
\end{lstlisting}

To fix this, prefix with \texttt{var}:

\begin{lstlisting}
var a = 5
\end{lstlisting}

\hypertarget{mtrxincmptbl}{%
\subsection{MtrxIncmptbl}\label{mtrxincmptbl}}

This error occurs when an arithmetic operation is performed on two
`incompatible' matrices. For example, two matrices must have the same
dimensions, i.e.~the same number of rows and columns, to be added or
subtracted,

\begin{lstlisting}
var a = Matrix([[1,2],[3,4]])
var b = Matrix([[1]])
print a+b // generates a `MtrxIncmptbl` error.
\end{lstlisting}

Or to be multiplied together, the number of columns of the left hand
matrix must equal the number of rows of the right hand matrix.

\begin{lstlisting}
var a = Matrix([[1,2],[3,4]])
var b = Matrix([1,2])
print a*b // ok
print b*a // generates a `MtrxIncmptbl` error.
\end{lstlisting}
