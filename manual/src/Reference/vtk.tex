\hypertarget{vtk}{%
\section{VTK}\label{vtk}}

The vtk module contains classes to allow I/O of meshes and fields using
the VTK Legacy Format.

\hypertarget{vtkexporter}{%
\section{VTKExporter}\label{vtkexporter}}

This class can be used to export the field(s) and/or at a given state to
a single .vtk file. To use it, import the \texttt{vtk} module:

\begin{lstlisting}
import vtk
\end{lstlisting}

Initialize the \texttt{VTKExporter}

\begin{lstlisting}
var vtkE = VTKExporter(obj)
\end{lstlisting}

where \texttt{obj} can either be

\begin{itemize}

\item
  A \texttt{Mesh} object: This prepares the Mesh for exporting.
\item
  A \texttt{Field} object: This prepares both the Field and the Mesh
  associated with it for exporting.
\end{itemize}

Use the \texttt{export} method to export to a VTK file.

\begin{lstlisting}
vtkE.export("output.vtk")
\end{lstlisting}

Optionally, use the \texttt{addfield} method to add one or more fields
before exporting:

\begin{lstlisting}
vtkE.addfield(f, fieldname="f")
\end{lstlisting}

where,

\begin{itemize}

\item
  \texttt{f} is the field object to be exported
\item
  \texttt{fieldname} is an optional argument that assigns a name to the
  field in the VTK file. This name is required to be a character string
  without embedded whitespace. If not provided, the name would be either
  ``scalars'' or ``vectors'' depending on the field type**.
\end{itemize}

** Note that this currently only supports scalar or vector (column
matrix) fields that live on the vertices ( shape \texttt{{[}1,0,0{]}}).
Support for tensorial fields and fields on cells coming soon.

Minimal example:

\begin{lstlisting}
import vtk
import meshtools

var m1 = LineMesh(fn (t) [t,0,0], -1..1:2)

var vtkE = VTKExporter(m1) // Export just the mesh 

vtkE.export("mesh.vtk")

var f1 = Field(m1, fn(x,y,z) x)

var g1 = Field(m1, fn(x,y,z) Matrix([x,2*x,3*x]))

vtkE = VTKExporter(f1, fieldname="f") // Export fields

vtkE.addfield(g1, fieldname="g")

vtkE.export("data.vtk")
\end{lstlisting}

\hypertarget{vtkimporter}{%
\section{VTKImporter}\label{vtkimporter}}

This class can be used to import the field(s) and/or the mesh at a given
state from a single .vtk file. To use it, import the \texttt{vtk}
module:

\begin{lstlisting}
import vtk
\end{lstlisting}

Initialize the \texttt{VTKImporter} with the filename

\begin{lstlisting}
var vtkI = VTKImporter("output.vtk")
\end{lstlisting}

Use the \texttt{mesh} method to get the mesh:

\begin{lstlisting}
var mesh = vtkI.mesh()
\end{lstlisting}

Use the \texttt{field} method to get the field:

\begin{lstlisting}
var f = vtkI.field(fieldname)
\end{lstlisting}

Use the \texttt{fieldlist} method to get the list of the names of the
fields contained in the file:

\begin{lstlisting}
print vtkI.fieldlist()
\end{lstlisting}

Use the \texttt{containsfield} method to check whether the file contains
a field by a given \texttt{fieldname}:

\begin{lstlisting}
if (tkI.containsfield(fieldname)) {
    ... 
}
\end{lstlisting}

where \texttt{fieldname} is the name assigned to the field in the .vtk
file

Minimal example:

\begin{lstlisting}
import vtk
import meshtools 

var vtkI = VTKImporter("data.vtk")

var m = vtkI.getmesh()

var f = vtkI.getfield("f")

var g = vtkI.getfield("g")
\end{lstlisting}
