\hypertarget{meshslice}{%
\section{Meshslice}\label{meshslice}}

The \texttt{meshslice} module is used to slice a \texttt{Mesh} object
along a given plane, yielding a new \texttt{Mesh} object of lower
dimensionality. You can also use \texttt{meshslice} to project
\texttt{Field} objects onto the new mesh.

To use the module, begin by importing it:

\begin{lstlisting}
import meshslice 
\end{lstlisting}

Then construct a \texttt{MeshSlicer} object, passing the mesh you want
to slice in the constructor:

\begin{lstlisting}
var slice = MeshSlicer(mesh)
\end{lstlisting}

You then perform a slice by calling the \texttt{slice} method, passing
the plane you want to slice through. This method returns a new
\texttt{Mesh} object comprising the slice. A plane is defined by a point
that lies on the plane \texttt{pt} and a direction normal to the plan
\texttt{dirn}:

\begin{lstlisting}
var slc = slice.slice(pt, dirn)
\end{lstlisting}

Having performed a slice, you can then project any associated
\texttt{Field} objects onto the sliced mesh by calling the
\texttt{slicefield} method:

\begin{lstlisting}
var phi = Field(mesh, fn (x,y,z) x+y+z)
var sphi = slice.slicefield(phi)
\end{lstlisting}

The new field returned by \texttt{slicefield} lives on the sliced mesh.
You can slice any number of fields.

You can perform multiple slices with the same \texttt{MeshSlicer} simply
by calling \texttt{slice} again with a different plane.

\hypertarget{slcempty}{%
\subsection{SlcEmpty}\label{slcempty}}

This error occurs if you try to use \texttt{slicefield} on a
\texttt{MeshSlicer} without having performed a slice. For example:

\begin{lstlisting}
var slice = MeshSlicer(mesh)
slice.slicefield(phi) // Throws SlcEmpty
slice.slice([0,0,0],[1,0,0]) 
\end{lstlisting}

To fix, call \texttt{slice} before \texttt{slicefield}:

\begin{lstlisting}
var slice = MeshSlicer(mesh)
slice.slice([0,0,0],[1,0,0]) 
slice.slicefield(phi) // Now slices correctly 
\end{lstlisting}
