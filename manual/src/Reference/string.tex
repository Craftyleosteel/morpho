\hypertarget{string}{%
\section{String}\label{string}}

Strings represent textual information. They are written in Morpho like
this:

\begin{lstlisting}
var a = "hello world"
\end{lstlisting}

Unicode characters including emoji are supported.

You can also create strings using the constructor function
\texttt{String}, which takes any number of parameters:

\begin{lstlisting}
var a = String("Hello", "World")
\end{lstlisting}

A very useful feature, called \emph{string interpolation}, enables the
results of any morpho expression can be interpolated into a string.
Here, the values of \texttt{i} and \texttt{func(i)} will be inserted
into the string as it is created:

\begin{lstlisting}
print "${i}: ${func(i)}"
\end{lstlisting}

To get an individual character, use index notatation

\begin{lstlisting}
print "morpho"[0]
\end{lstlisting}

You can loop over each character like this:

\begin{lstlisting}
for (c in "morpho") print c
\end{lstlisting}

Note that strings are immutable, and hence

\begin{lstlisting}
var a = "morpho"
a[0] = 4
\end{lstlisting}

raises an error.

\hypertarget{split}{%
\subsection{split}\label{split}}

The split method splits a String into a list of substrings. It takes one
argument, which is a string of characters to use to split the string:

\begin{lstlisting}
print "1,2,3".split(",")
\end{lstlisting}

gives

\begin{lstlisting}
[ 1, 2, 3 ]
\end{lstlisting}
