\hypertarget{graphics}{%
\section{Graphics}\label{graphics}}

The \texttt{graphics} module provides a number of classes to provide
simple visualization capabilities. To use it, you first need to import
the module:

\begin{lstlisting}
import graphics
\end{lstlisting}

The \texttt{Graphics} class acts as an abstract container for graphical
information; to actually launch the display see the \texttt{Show} class.
You can create an empty scene like this,

\begin{lstlisting}
var g = Graphics()
\end{lstlisting}

Additional elements can be added using the \texttt{display} method.

\begin{lstlisting}
g.display(element)
\end{lstlisting}

Morpho provides the following fundamental Graphical element classes:

\begin{lstlisting}
TriangleComplex
\end{lstlisting}

You can also use functions like \texttt{Arrow}, \texttt{Tube} and
\texttt{Cylinder} to create these elements conveniently.

To combine graphics objects, use the add operator:

\begin{lstlisting}
var g1 = Graphics(), g2 = Graphics()
// ...
Show(g1+g2)
\end{lstlisting}

\hypertarget{show}{%
\subsection{Show}\label{show}}

\texttt{Show} is used to launch an interactive graphical display using
the external \texttt{morphoview} application. \texttt{Show} takes a
\texttt{Graphics} object as an argument:

\begin{lstlisting}
var g = Graphics()
Show(g)
\end{lstlisting}

\hypertarget{trianglecomplex}{%
\subsection{TriangleComplex}\label{trianglecomplex}}

A \texttt{TriangleComplex} is a graphical element that can be used as
part of a graphical display. It consists of a list of vertices and a
connectivity matrix that selects which vertices are used in each
triangle.

To create one, call the constructor with the following arguments:

\begin{lstlisting}
TriangleComplex(position, normals, colors, connectivity)
\end{lstlisting}

\begin{itemize}

\item
  \texttt{position} is a \texttt{Matrix} containing vertex positions as
  \emph{columns}.
\item
  \texttt{normals} is a \texttt{Matrix} with a normal for each vertex.
\item
  \texttt{colors} is the color of the object.
\item
  \texttt{connectivity} is a \texttt{Sparse} matrix where each column
  represents a triangle and rows correspond to vertices.
\end{itemize}

You can also provide optional arguments:

\begin{itemize}

\item
  \texttt{transmit} sets the transparency of the object. This parameter
  is only used by the povray module as of now. Default is 0.
\item
  \texttt{filter} sets the transparency of the object using a filter
  effect. This parameter is only used by the povray module as of now.
  Default is 0. For the difference between \texttt{transmit} and
  \texttt{filter}, checkout the
  \href{http://xahlee.info/3d/povray-glassy.html}{POVRay documentation}.
\end{itemize}

Add to a \texttt{Graphics} object using the \texttt{display} method.

\hypertarget{arrow}{%
\subsection{Arrow}\label{arrow}}

The \texttt{Arrow} function creates an arrow. It takes two arguments:

\begin{lstlisting}
arrow(start, end)
\end{lstlisting}

\begin{itemize}

\item
  \texttt{start} and \texttt{end} are the two vertices. The arrow points
  \texttt{start} -\textgreater{} \texttt{end}.
\end{itemize}

You can also provide optional arguments:

\begin{itemize}

\item
  \texttt{aspectratio} controls the width of the arrow relative to its
  length
\item
  \texttt{n} is an integer that controls the quality of the display.
  Higher \texttt{n} leads to a rounder arrow.
\item
  \texttt{color} is the color of the arrow. This can be a list of RGB
  values or a \texttt{Color} object
\item
  \texttt{transmit} sets the transparency of the arrow. This parameter
  is only used by the povray module as of now. Default is 0.
\item
  \texttt{filter} sets the transparency of the arrow using a filter
  effect. This parameter is only used by the povray module as of now.
  Default is 0. For the difference between \texttt{transmit} and
  \texttt{filter}, checkout the
  \href{http://xahlee.info/3d/povray-glassy.html}{POVRay documentation}.
\end{itemize}

Display an arrow:

\begin{lstlisting}
var g = Graphics([])
g.display(Arrow([-1/2,-1/2,-1/2], [1/2,1/2,1/2], aspectratio=0.05, n=10))
Show(g)
\end{lstlisting}

\hypertarget{cylinder}{%
\subsection{Cylinder}\label{cylinder}}

The \texttt{Cylinder} function creates a cylinder. It takes two required
arguments:

\begin{lstlisting}
cylinder(start, end)
\end{lstlisting}

\begin{itemize}

\item
  \texttt{start} and \texttt{end} are the two vertices.
\end{itemize}

You can also provide optional arguments:

\begin{itemize}

\item
  \texttt{aspectratio} controls the width of the cylinder relative to
  its length.
\item
  \texttt{n} is an integer that controls the quality of the display.
  Higher \texttt{n} leads to a rounder cylinder.
\item
  \texttt{color} is the color of the cylinder. This can be a list of RGB
  values or a \texttt{Color} object.
\item
  \texttt{transmit} sets the transparency of the cylinder. This
  parameter is only used by the povray module as of now. Default is 0.
\item
  \texttt{filter} sets the transparency of the cylinder using a filter
  effect. This parameter is only used by the povray module as of now.
  Default is 0. For the difference between \texttt{transmit} and
  \texttt{filter}, checkout the
  \href{http://xahlee.info/3d/povray-glassy.html}{POVRay documentation}.
\end{itemize}

Display an cylinder:

\begin{lstlisting}
var g = Graphics()
g.display(Cylinder([-1/2,-1/2,-1/2], [1/2,1/2,1/2], aspectratio=0.1, n=10))
Show(g)
\end{lstlisting}

\hypertarget{tube}{%
\subsection{Tube}\label{tube}}

The \texttt{Tube} function connects a sequence of points to form a tube.

\begin{lstlisting}
Tube(points, radius)
\end{lstlisting}

\begin{itemize}

\item
  \texttt{points} is a list of points; this can be a list of lists or a
  \texttt{Matrix} with the positions as columns.
\item
  \texttt{radius} is the radius of the tube.
\end{itemize}

You can also provide optional arguments:

\begin{itemize}

\item
  \texttt{n} is an integer that controls the quality of the display.
  Higher \texttt{n} leads to a rounder tube.
\item
  \texttt{color} is the color of the tube. This can be a list of RGB
  values or a \texttt{Color} object.
\item
  \texttt{closed} is a \texttt{bool} that indicates whether the tube
  should be closed to form a loop.
\item
  \texttt{transmit} sets the transparency of the tube. This parameter is
  only used by the povray module as of now. Default is 0.
\item
  \texttt{filter} sets the transparency of the tube using a filter
  effect. This parameter is only used by the povray module as of now.
  Default is 0. For the difference between \texttt{transmit} and
  \texttt{filter}, checkout the
  \href{http://xahlee.info/3d/povray-glassy.html}{POVRay documentation}.
\end{itemize}

Draw a square:

\begin{lstlisting}
var a = Tube([[-1/2,-1/2,0],[1/2,-1/2,0],[1/2,1/2,0],[-1/2,1/2,0]], 0.1, closed=true)
var g = Graphics()
g.display(a)
\end{lstlisting}

\hypertarget{sphere}{%
\subsection{Sphere}\label{sphere}}

The \texttt{Sphere} function creates a sphere.

\begin{lstlisting}
Sphere(center, radius)
\end{lstlisting}

\begin{itemize}

\item
  \texttt{center} is the position of the center of the sphere; this can
  be a list or column \texttt{Matrix}.
\item
  \texttt{radius} is the radius of the sphere
\end{itemize}

You can also provide optional arguments:

\begin{itemize}

\item
  \texttt{color} is the color of the sphere. This can be a list of RGB
  values or a \texttt{Color} object.
\item
  \texttt{transmit} sets the transparency of the sphere. This parameter
  is only used by the povray module as of now. Default is 0.
\item
  \texttt{filter} sets the transparency of the sphere using a filter
  effect. This parameter is only used by the povray module as of now.
  Default is 0. For the difference between \texttt{transmit} and
  \texttt{filter}, checkout the
  \href{http://xahlee.info/3d/povray-glassy.html}{POVRay documentation}.
\end{itemize}

Draw some randomly sized spheres:

\begin{lstlisting}
var g = Graphics()
for (i in 0...10) {
  g.display(Sphere([random()-1/2, random()-1/2, random()-1/2], 0.1*(1+random()),       color=Gray(random())))
}
Show(g)
\end{lstlisting}

\hypertarget{text}{%
\subsection{Text}\label{text}}

A \texttt{Text} object is used to display text.

\begin{lstlisting}
Text(text, position)
\end{lstlisting}

\begin{itemize}

\item
  \texttt{text} is the text to display as a string.
\item
  \texttt{position} is the position at which to display the text.
\end{itemize}

You can also provide optional arguments:

\begin{itemize}

\item
  \texttt{color} is the color of the text. This should be a
  \texttt{Color} object.
\item
  \texttt{dirn} is the direction along which the text is drawn. This
  should be a \texttt{List} or a \texttt{Matrix}.
\item
  \texttt{size} is the font size to use
\item
  \texttt{vertical} is the vertical direction for the text
\item
  \texttt{font} is the \texttt{Font} object to use.
\end{itemize}

Draw several pieces of text around the y axis:

\begin{lstlisting}
var g = Graphics()
for (phi in 0..Pi:Pi/8) {
  g.display(Text("Hello World", [0,0,0], size=72, dirn=[0,1,0], vertical=[cos(phi),0,sin(phi)]))
}
Show(g)
\end{lstlisting}
