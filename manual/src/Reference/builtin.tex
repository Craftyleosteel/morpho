\hypertarget{builtin-functions}{%
\section{Builtin functions}\label{builtin-functions}}

Morpho provides a number of built-in functions.

\hypertarget{sin}{%
\subsection{sin}\label{sin}}

Returns the sine of an angle in radians.

\begin{lstlisting}
import constants 
print sin(Pi/2) // expect: 1
\end{lstlisting}

\hypertarget{sinh}{%
\subsection{sinh}\label{sinh}}

Returns the hyperbolic sine of an angle in radians.

\begin{lstlisting}
import constants 
print sinh(0) // expect: 0
\end{lstlisting}

\hypertarget{cos}{%
\subsection{cos}\label{cos}}

Returns the cosine of an angle in radians.

\begin{lstlisting}
import constants 
print cos(Pi/2) // expect: 0
\end{lstlisting}

\hypertarget{cosh}{%
\subsection{cosh}\label{cosh}}

Returns the hyperbolic cosine of an angle in radians.

\begin{lstlisting}
print cosh(0) // expect: 1
\end{lstlisting}

\hypertarget{arctan}{%
\subsection{arctan}\label{arctan}}

Returns the arctangent of an input value that lies from \texttt{-Inf} to
\texttt{Inf}. You can use one argument:

\begin{lstlisting}
print arctan(0) // expect: 0
\end{lstlisting}

or use two arguments to return the angle in the correct quadrant:

\begin{lstlisting}
print arctan(x, y)
\end{lstlisting}

Note the order \texttt{x}, \texttt{y} differs from some other languages.

\hypertarget{isnil}{%
\subsection{isnil}\label{isnil}}

Returns \texttt{true} if a value is \texttt{nil} or \texttt{false}
otherwise.

\hypertarget{isint}{%
\subsection{isint}\label{isint}}

Returns \texttt{true} if a value is an integer or \texttt{false}
otherwise.

\hypertarget{isfloat}{%
\subsection{isfloat}\label{isfloat}}

Returns \texttt{true} if a value is a floating point number or
\texttt{false} otherwise.

\hypertarget{isbool}{%
\subsection{isbool}\label{isbool}}

Returns \texttt{true} if a value is a boolean or \texttt{false}
otherwise.

\hypertarget{isobject}{%
\subsection{isobject}\label{isobject}}

Returns \texttt{true} if a value is an object or \texttt{false}
otherwise.

\hypertarget{isstring}{%
\subsection{isstring}\label{isstring}}

Returns \texttt{true} if a value is a string or \texttt{false}
otherwise.

\hypertarget{isclass}{%
\subsection{isclass}\label{isclass}}

Returns \texttt{true} if a value is a class or \texttt{false} otherwise.

\hypertarget{isrange}{%
\subsection{isrange}\label{isrange}}

Returns \texttt{true} if a value is a range or \texttt{false} otherwise.

\hypertarget{isdictionary}{%
\subsection{isdictionary}\label{isdictionary}}

Returns \texttt{true} if a value is a dictionary or \texttt{false}
otherwise.

\hypertarget{islist}{%
\subsection{islist}\label{islist}}

Returns \texttt{true} if a value is a list or \texttt{false} otherwise.

\hypertarget{isarray}{%
\subsection{isarray}\label{isarray}}

Returns \texttt{true} if a value is an array or \texttt{false}
otherwise.

\hypertarget{ismatrix}{%
\subsection{ismatrix}\label{ismatrix}}

Returns \texttt{true} if a value is a matrix or \texttt{false}
otherwise.

\hypertarget{issparse}{%
\subsection{issparse}\label{issparse}}

Returns \texttt{true} if a value is a sparse matrix or \texttt{false}
otherwise.

\hypertarget{isinf}{%
\subsection{isinf}\label{isinf}}

Returns \texttt{true} if a value is infinite or \texttt{false}
otherwise.

\hypertarget{isnan}{%
\subsection{isnan}\label{isnan}}

Returns \texttt{true} if a value is a Not a Number or \texttt{false}
otherwise.

\hypertarget{iscallable}{%
\subsection{iscallable}\label{iscallable}}

Returns \texttt{true} if a value is callable or \texttt{false}
otherwise.

\hypertarget{apply}{%
\subsection{Apply}\label{apply}}

Apply calls a function with the arguments provided as a list:

\begin{lstlisting}
apply(f, [0.5, 0.5]) // calls f(0.5, 0.5) 
\end{lstlisting}

It's often useful where a function or method and/or the number of
parameters isn't known ahead of time. The first parameter to apply can
be any callable object, including a method invocation or a closure.

You may also instead omit the list and use apply with multiple
arguments:

\begin{lstlisting}
apply(f, 0.5, 0.5) // calls f(0.5, 0.5)
\end{lstlisting}

There is one edge case that occurs when you want to call a function that
accepts a single list as a parameter. In this case, enclose the list in
another list:

\begin{lstlisting}
apply(f, [[1,2]]) // equivalent to f([1,2])
\end{lstlisting}
