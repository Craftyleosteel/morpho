\hypertarget{builtin-functions}{%
\section{Builtin functions}\label{builtin-functions}}

Morpho provides a number of built-in functions.

\hypertarget{random}{%
\subsection{Random}\label{random}}

The \texttt{random} function generates a random number from a uniform
distribution on the interval {[}0,1{]}.

\begin{lstlisting}
print random() 
\end{lstlisting}

See also \texttt{randomnormal} and \texttt{randomint}.

\hypertarget{randomnormal}{%
\subsection{Randomnormal}\label{randomnormal}}

The \texttt{randomnormal} function generates a random number from a
normal (gaussian) distribution with unit variance and zero offset.

\begin{lstlisting}
print randomnormal() 
\end{lstlisting}

See also \texttt{random} and \texttt{randomint}.

\hypertarget{randomint}{%
\subsection{Randomint}\label{randomint}}

The \texttt{randomint} function generates a random integer with a
specified maximum value.

\begin{lstlisting}
print randomint(10) // Generates a random integer [0,10)
\end{lstlisting}

\hypertarget{isnil}{%
\subsection{isnil}\label{isnil}}

Returns \texttt{true} if a value is \texttt{nil} or \texttt{false}
otherwise.

\hypertarget{isint}{%
\subsection{isint}\label{isint}}

Returns \texttt{true} if a value is an integer or \texttt{false}
otherwise.

\hypertarget{isfloat}{%
\subsection{isfloat}\label{isfloat}}

Returns \texttt{true} if a value is a floating point number or
\texttt{false} otherwise.

\hypertarget{isbool}{%
\subsection{isbool}\label{isbool}}

Returns \texttt{true} if a value is a boolean or \texttt{false}
otherwise.

\hypertarget{isobject}{%
\subsection{isobject}\label{isobject}}

Returns \texttt{true} if a value is an object or \texttt{false}
otherwise.

\hypertarget{isstring}{%
\subsection{isstring}\label{isstring}}

Returns \texttt{true} if a value is a string or \texttt{false}
otherwise.

\hypertarget{isclass}{%
\subsection{isclass}\label{isclass}}

Returns \texttt{true} if a value is a class or \texttt{false} otherwise.

\hypertarget{isrange}{%
\subsection{isrange}\label{isrange}}

Returns \texttt{true} if a value is a range or \texttt{false} otherwise.

\hypertarget{isdictionary}{%
\subsection{isdictionary}\label{isdictionary}}

Returns \texttt{true} if a value is a dictionary or \texttt{false}
otherwise.

\hypertarget{islist}{%
\subsection{islist}\label{islist}}

Returns \texttt{true} if a value is a list or \texttt{false} otherwise.

\hypertarget{isarray}{%
\subsection{isarray}\label{isarray}}

Returns \texttt{true} if a value is an array or \texttt{false}
otherwise.

\hypertarget{ismatrix}{%
\subsection{ismatrix}\label{ismatrix}}

Returns \texttt{true} if a value is a matrix or \texttt{false}
otherwise.

\hypertarget{issparse}{%
\subsection{issparse}\label{issparse}}

Returns \texttt{true} if a value is a sparse matrix or \texttt{false}
otherwise.

\hypertarget{isinf}{%
\subsection{isinf}\label{isinf}}

Returns \texttt{true} if a value is infinite or \texttt{false}
otherwise.

\hypertarget{isnan}{%
\subsection{isnan}\label{isnan}}

Returns \texttt{true} if a value is a Not a Number or \texttt{false}
otherwise.

\hypertarget{iscallable}{%
\subsection{iscallable}\label{iscallable}}

Returns \texttt{true} if a value is callable or \texttt{false}
otherwise.

\hypertarget{isfinite}{%
\subsection{isfinite}\label{isfinite}}

Returns \texttt{true} if a value is finite or \texttt{false} otherwise.

\begin{lstlisting}
print isfinite(1) // expect: true 
print isfinite(1/0) // expect: false 
\end{lstlisting}

\hypertarget{isnumber}{%
\subsection{isnumber}\label{isnumber}}

Returns \texttt{true} if a value is a real number, or \texttt{false}
otherwise.

\begin{lstlisting}
print isnumber(1) // expect: true 
print isnumber(Object()) // expect: false
\end{lstlisting}

\hypertarget{ismesh}{%
\subsection{ismesh}\label{ismesh}}

Returns \texttt{true} if a value is a \texttt{Mesh}, or \texttt{false}
otherwise.

\hypertarget{isselection}{%
\subsection{isselection}\label{isselection}}

Returns \texttt{true} if a value is a \texttt{Selection}, or
\texttt{false} otherwise.

\hypertarget{isfield}{%
\subsection{isfield}\label{isfield}}

Returns \texttt{true} if a value is a \texttt{Field}, or \texttt{false}
otherwise.

\hypertarget{apply}{%
\subsection{Apply}\label{apply}}

Apply calls a function with the arguments provided as a list:

\begin{lstlisting}
apply(f, [0.5, 0.5]) // calls f(0.5, 0.5) 
\end{lstlisting}

It's often useful where a function or method and/or the number of
parameters isn't known ahead of time. The first parameter to apply can
be any callable object, including a method invocation or a closure.

You may also instead omit the list and use apply with multiple
arguments:

\begin{lstlisting}
apply(f, 0.5, 0.5) // calls f(0.5, 0.5)
\end{lstlisting}

There is one edge case that occurs when you want to call a function that
accepts a single list as a parameter. In this case, enclose the list in
another list:

\begin{lstlisting}
apply(f, [[1,2]]) // equivalent to f([1,2])
\end{lstlisting}

\hypertarget{abs}{%
\subsection{Abs}\label{abs}}

Returns the absolute value of a number:

\begin{lstlisting}
print abs(-10) // prints 10 
\end{lstlisting}

\hypertarget{arctan}{%
\subsection{Arctan}\label{arctan}}

Returns the arctangent of an input value that lies from \texttt{-Inf} to
\texttt{Inf}. You can use one argument:

\begin{lstlisting}
print arctan(0) // expect: 0
\end{lstlisting}

or use two arguments to return the angle in the correct quadrant:

\begin{lstlisting}
print arctan(x, y)
\end{lstlisting}

Note the order \texttt{x}, \texttt{y} differs from some other languages.

\hypertarget{exp}{%
\subsection{Exp}\label{exp}}

Exponential function \texttt{e\^{}x}. Inverse of \texttt{log}.

\begin{lstlisting}
print exp(0) // expect: 1 
print exp(Pi*im) // expect: -1 + 0im
\end{lstlisting}

\hypertarget{log}{%
\subsection{Log}\label{log}}

Natural logarithm function. Inverse of \texttt{exp}.

\begin{lstlisting}
print log(1) // expect: 0 
\end{lstlisting}

\hypertarget{log10}{%
\subsection{Log10}\label{log10}}

Base 10 logarithm function.

\begin{lstlisting}
print log10(10) // expect: 1
\end{lstlisting}

\hypertarget{sin}{%
\subsection{Sin}\label{sin}}

Sine trigonometric function.

\begin{lstlisting}
print sin(0) // expect: 0 
\end{lstlisting}

\hypertarget{sinh}{%
\subsection{Sinh}\label{sinh}}

Hyperbolic sine trigonometric function.

\begin{lstlisting}
print sinh(0) // expect: 0 
\end{lstlisting}

\hypertarget{cos}{%
\subsection{Cos}\label{cos}}

Cosine trigonometric function.

\begin{lstlisting}
print cos(0) // expect: 1
\end{lstlisting}

\hypertarget{cosh}{%
\subsection{Cosh}\label{cosh}}

Hyperbolic cosine trigonometric function.

\begin{lstlisting}
print cosh(0) // expect: 1
\end{lstlisting}

\hypertarget{tan}{%
\subsection{Tan}\label{tan}}

Tangent trigonometric function.

\begin{lstlisting}
print tan(0) // expect: 0 
\end{lstlisting}

\hypertarget{tanh}{%
\subsection{Tanh}\label{tanh}}

Hyperbolic tangent trigonometric function.

\begin{lstlisting}
print tanh(0) // expect: 0 
\end{lstlisting}

\hypertarget{asin}{%
\subsection{Asin}\label{asin}}

Inverse sine trigonometric function. Returns a value on the interval
\texttt{{[}-Pi/2,Pi/2{]}}.

\begin{lstlisting}
print asin(0) // expect: 0 
\end{lstlisting}

\hypertarget{acos}{%
\subsection{Acos}\label{acos}}

Inverse cosine trigonometric function. Returns a value on the interval
\texttt{{[}-Pi/2,Pi/2{]}}.

\begin{lstlisting}
print acos(1) // expect: 0 
\end{lstlisting}

\hypertarget{sqrt}{%
\subsection{Sqrt}\label{sqrt}}

Square root function.

\begin{lstlisting}
print sqrt(4) // expect: 2
\end{lstlisting}

\hypertarget{min}{%
\subsection{Min}\label{min}}

Finds the minimum value of its arguments. If any of the arguments are
Objects and are enumerable, (e.g.~a \texttt{List}), \texttt{min} will
search inside them for a minimum value. Accepts any number of arguments.

\begin{lstlisting}
print min(3,2,1) // expect: 1 
print min([3,2,1]) // expect: 1 
print min([3,2,1],[0,-1,2]) // expect: -2 
\end{lstlisting}

\hypertarget{max}{%
\subsection{Max}\label{max}}

Finds the maximum value of its arguments. If any of the arguments are
Objects and are enumerable, (e.g.~a \texttt{List}), \texttt{max} will
search inside them for a maximum value. Accepts any number of arguments.

\begin{lstlisting}
print min(3,2,1) // expect: 3 
print min([3,2,1]) // expect: 3
print min([3,2,1],[0,-1,2]) // expect: 3 
\end{lstlisting}

\hypertarget{bounds}{%
\subsection{Bounds}\label{bounds}}

Returns both the results of \texttt{min} and \texttt{max} as a list,
Providing a set of bounds for its arguments and any enumerable objects
within them.

\begin{lstlisting}
print bounds(1,2,3) // expect: [1,3]
print bounds([3,2,1],[0,-1,2]) // expect: [-1,3]
\end{lstlisting}
