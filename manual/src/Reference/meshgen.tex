\hypertarget{meshgen}{%
\section{Meshgen}\label{meshgen}}

The \texttt{meshgen} module is used to create \texttt{Mesh} objects
corresponding to a specified domain. It provides the \texttt{MeshGen}
class to perform the meshing, which are created with the following
arguments:

\begin{lstlisting}
MeshGen(domain, boundingbox)
\end{lstlisting}

Domains are specified by a scalar function that is positive in the
region to be meshed and locally smooth. For example, to mesh the unit
disk:

\begin{lstlisting}
var dom = fn (x) -(x[0]^2+x[1]^2-1)
\end{lstlisting}

A \texttt{MeshGen} object is then created and then used to build the
\texttt{Mesh} like this:

\begin{lstlisting}
var mg = MeshGen(dom, [-1..1:0.2, -1..1:0.2])
var m = mg.build()
\end{lstlisting}

A bounding box for the mesh must be specified as a \texttt{List} of
\texttt{Range} objects, one for each dimension. The increment on each
\texttt{Range} gives an approximate scale for the size of elements
generated.

To facilitate convenient creation of domains, a \texttt{Domain} class is
provided that provides set operations \texttt{union},
\texttt{intersection} and \texttt{difference}.

\texttt{MeshGen} accepts a number of optional arguments:

\begin{itemize}

\item
  \texttt{weight} A scalar weight function that controls mesh density.
\item
  \texttt{quiet} Set to \texttt{true} to suppress \texttt{MeshGen}
  output.
\item
  \texttt{method} a list of options that controls the method used.
\end{itemize}

Some method choices that are available include:

\begin{itemize}

\item
  \texttt{"FixedStepSize"} Use a fixed step size in optimization.
\item
  \texttt{"StartGrid"} Start from a regular grid of points (the
  default).
\item
  \texttt{"StartRandom"} Start from a randomly generated collection of
  points.
\end{itemize}

There are also a number of properties of a \texttt{MeshGen} object that
can be set prior to calling \texttt{build} to control the operation of
the mesh generation:

\begin{itemize}

\item
  \texttt{stepsize}, \texttt{steplimit} Stepsize used internally by the
  \texttt{Optimizer}
\item
  \texttt{fscale} an internal ``pressure''
\item
  \texttt{ttol} how far the vertices are allowed to move before
  retriangulation
\item
  \texttt{etol} energy tolerance for optimization problem
\end{itemize}

\texttt{MeshGen} picks default values that cover a reasonable range of
uses.

\hypertarget{domain}{%
\subsection{Domain}\label{domain}}

The \texttt{Domain} class is used to conveniently build a domain by
composing simpler elements.

Create a \texttt{Domain} from a scalar function that is positive in the
region of interest:

\begin{lstlisting}
var dom = Domain(fn (x) -(x[0]^2+x[1]^2-1))
\end{lstlisting}

You can pass it to \texttt{MeshGen} to specify the region to mesh:

\begin{lstlisting}
var mg = MeshGen(dom, [-1..1:0.2, -1..1:0.2])
\end{lstlisting}

You can combine \texttt{Domain} objects using set operations
\texttt{union}, \texttt{intersection} and \texttt{difference}:

\begin{lstlisting}
var a = CircularDomain(Matrix([-0.5,0]), 1)
var b = CircularDomain(Matrix([0.5,0]), 1)
var c = CircularDomain(Matrix([0,0]), 0.3)
var dom = a.union(b).difference(c)
\end{lstlisting}

\hypertarget{circulardomain}{%
\subsection{CircularDomain}\label{circulardomain}}

Conveniently constructs a \texttt{Domain} object correspondiong to a
disk. Requires the position of the center and a radius as arguments.

Create a domain corresponding to the unit disk:

\begin{lstlisting}
var c = CircularDomain([0,0], 1)
\end{lstlisting}

\hypertarget{halfspacedomain}{%
\subsection{HalfSpaceDomain}\label{halfspacedomain}}

Conveniently constructs a \texttt{Domain} object correspondiong to a
half space defined by a plane at \texttt{x0} and a normal \texttt{n}:

\begin{lstlisting}
var hs = HalfSpaceDomain(x0, n)
\end{lstlisting}

Note \texttt{n} is an ``outward'' normal, so points into the
\emph{excluded} region.

Half space corresponding to the allowed region \texttt{x\textless{}0}:

\begin{lstlisting}
var hs = HalfSpaceDomain(Matrix([0,0,0]), Matrix([1,0,0]))
\end{lstlisting}

Note that \texttt{HalfSpaceDomain}s cannot be meshed directly as they
correspond to an infinite region. They are useful, however, for
combining with other domains.

Create half a disk by cutting a \texttt{HalfSpaceDomain} from a
\texttt{CircularDomain}:

\begin{lstlisting}
var c = CircularDomain([0,0], 1)
var hs = HalfSpaceDomain(Matrix([0,0]), Matrix([-1,0]))
var dom = c.difference(hs) 
var mg = MeshGen(dom, [-1..1:0.2, -1..1:0.2], quiet=false)
var m = mg.build()
\end{lstlisting}

\hypertarget{mshgndim}{%
\subsection{MshGnDim}\label{mshgndim}}

The \texttt{MeshGen} module currently supports 2 and 3 dimensional
meshes. Higher dimensional meshing will be available in a future
release; please contact the developer if you are interested in this
functionality.
