\hypertarget{optimize}{%
\section{Optimize}\label{optimize}}

The \texttt{optimize} package contains a number of functions and classes
to perform shape optimization.

\hypertarget{optimizationproblem}{%
\subsection{OptimizationProblem}\label{optimizationproblem}}

An \texttt{OptimizationProblem} object defines an optimization problem,
which may include functionals to optimize as well as global and local
constraints.

Create an \texttt{OptimizationProblem} with a mesh:

\begin{lstlisting}
var problem = OptimizationProblem(mesh)
\end{lstlisting}

Add an energy:

\begin{lstlisting}
var la = Area()
problem.addenergy(la)
\end{lstlisting}

Add an energy that operates on a selected region, and with an optional
prefactor:

\begin{lstlisting}
problem.addenergy(la, selection=sel, prefactor=2)
\end{lstlisting}

Add a constraint:

\begin{lstlisting}
problem.addconstraint(la)
\end{lstlisting}

Add a local constraint (here a onesided level set constraint):

\begin{lstlisting}
var ls = ScalarPotential(fn (x,y,z) z, fn (x,y,z) Matrix([0,0,1]))
problem.addlocalconstraint(ls, onesided=true)
\end{lstlisting}

\hypertarget{optimizer}{%
\subsection{Optimizer}\label{optimizer}}

\texttt{Optimizer} objects are used to optimize \texttt{Mesh}es and
\texttt{Field}s. You should use the appropriate subclass:
\texttt{ShapeOptimizer} or \texttt{FieldOptimizer} respectively.

\hypertarget{shapeoptimizer}{%
\subsection{ShapeOptimizer}\label{shapeoptimizer}}

A \texttt{ShapeOptimizer} object performs shape optimization: it moves
the vertex positions to reduce an overall energy.

Create a \texttt{ShapeOptimizer} object with an
\texttt{OptimizationProblem} and a \texttt{Mesh}:

\begin{lstlisting}
var sopt = ShapeOptimizer(problem, m)
\end{lstlisting}

Take a step down the gradient with fixed stepsize:

\begin{lstlisting}
sopt.relax(5) // Takes five steps
\end{lstlisting}

Linesearch down the gradient:

\begin{lstlisting}
sopt.linesearch(5) // Performs five linesearches
\end{lstlisting}

Perform conjugate gradient (usually gives faster convergence):

\begin{lstlisting}
sopt.conjugategradient(5) // Performs five conjugate gradient steps.
\end{lstlisting}

Control a number of properties of the optimizer:

\begin{lstlisting}
sopt.stepsize=0.1 // The stepsize to take
sopt.steplimit=0.5 // Maximum stepsize for optimizing methods
sopt.etol = 1e-8 // Energy convergence tolerance
sopt.ctol = 1e-9 // Tolerance to which constraints are satisfied
sopt.maxconstraintsteps = 20 // Maximum number of constraint steps to use
\end{lstlisting}

\hypertarget{fieldoptimizer}{%
\subsection{FieldOptimizer}\label{fieldoptimizer}}

A \texttt{FieldOptimizer} object performs field optimization: it changes
elements of a \texttt{Field} to reduce an overall energy.

Create a \texttt{FieldOptimizer} object with an
\texttt{OptimizationProblem} and a \texttt{Field}:

\begin{lstlisting}
var sopt = FieldOptimizer(problem, fld)
\end{lstlisting}

Field optimizers provide the same options and methods as Shape
optimizers: see the \texttt{ShapeOptimizer} documentation for details.
