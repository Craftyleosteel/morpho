\hypertarget{variables}{%
\section{Variables}\label{variables}}

Variables are defined using the \texttt{var} keyword followed by the
variable name:

\begin{lstlisting}
var a
\end{lstlisting}

Optionally, an initial assignment may be given:

\begin{lstlisting}
var a = 1
\end{lstlisting}

Variables defined in a block of code are visible only within that block,
so

\begin{lstlisting}
var greeting = "Hello"
{
    var greeting = "Goodbye"
    print greeting
}
print greeting
\end{lstlisting}

will print

\emph{Goodbye} \emph{Hello}

Multiple variables can be defined at once by separating them with commas

\begin{lstlisting}
var a, b=2, c[2]=[1,2]
\end{lstlisting}

where each can have its own initializer (or not).

\hypertarget{indexing}{%
\subsection{Indexing}\label{indexing}}

Morpho provides a number of collection objects, such as \texttt{List},
\texttt{Range}, \texttt{Array}, \texttt{Dictionary}, \texttt{Matrix} and
\texttt{Sparse}, that can contain more than one value. Index notation
(sometimes called subscript notation) is used to access elements of
these objects.

To retrieve an item from a collection, you use the \texttt{{[}} and
\texttt{{]}} brackets like this:

\begin{lstlisting}
var a = List("Apple", "Bag", "Cat")
print a[0]
\end{lstlisting}

which prints \emph{Apple}. Note that the first element is accessed with
\texttt{0} not \texttt{1}.

Similarly, to set an entry in a collection, use:

\begin{lstlisting}
a[0]="Adder"
\end{lstlisting}

which would replaces the first element in \texttt{a} with
\texttt{"Adder"}.

Some collection objects need more than one index,

\begin{lstlisting}
var a = Matrix([[1,0],[0,1]])
print a[0,0]
\end{lstlisting}

and others such as \texttt{Dictionary} use non-numerical indices,

\begin{lstlisting}
var b = Dictionary()
b["Massachusetts"]="Boston"
b["California"]="Sacramento"
\end{lstlisting}

as in this dictionary of state capitals.
