\hypertarget{functionals}{%
\section{Functionals}\label{functionals}}

A number of \texttt{functionals} are available in Morpho. Each of these
represents an integral over some \texttt{Mesh} and \texttt{Field}
objects (on a particular \texttt{Selection}) and are used to define
energies and constraints in an \texttt{OptimizationProblem} provided by
the \texttt{optimize} module.

Many functionals are built in. Additional functionals are available by
importing the \texttt{functionals} module:

\begin{lstlisting}
import functionals
\end{lstlisting}

Functionals provide a number of standard methods:

\begin{itemize}

\item
  \texttt{total}(mesh) - returns the value of the integral with a
  provided mesh, selection and fields
\item
  \texttt{integrand}(mesh) - returns the contribution to the integral
  from each element
\item
  \texttt{gradient}(mesh) - returns the gradient of the functional with
  respect to vertex motions.
\item
  \texttt{fieldgradient}(mesh, field) - returns the gradient of the
  functional with respect to components of the field
\end{itemize}

Each of these may be called with a mesh, a field and a selection.

\hypertarget{length}{%
\section{Length}\label{length}}

A \texttt{Length} functional calculates the length of a line element in
a mesh.

Evaluate the length of a circular loop:

\begin{lstlisting}
import constants
import meshtools
var m = LineMesh(fn (t) [cos(t), sin(t), 0], 0...2*Pi:Pi/20, closed=true)
var le = Length()
print le.total(m)
\end{lstlisting}

\hypertarget{areaenclosed}{%
\section{AreaEnclosed}\label{areaenclosed}}

An \texttt{AreaEnclosed} functional calculates the area enclosed by a
loop of line elements.

\begin{lstlisting}
var la = AreaEnclosed()
\end{lstlisting}

\hypertarget{area}{%
\section{Area}\label{area}}

An \texttt{Area} functional calculates the area of the area elements in
a mesh:

\begin{lstlisting}
var la = Area()
print la.total(mesh)
\end{lstlisting}

\hypertarget{volumeenclosed}{%
\section{VolumeEnclosed}\label{volumeenclosed}}

A \texttt{VolumeEnclosed} functional is used to calculate the volume
enclosed by a surface. Note that this estimate may become inaccurate for
highly deformed surfaces.

\begin{lstlisting}
var lv = VolumeEnclosed()
\end{lstlisting}

\hypertarget{volume}{%
\section{Volume}\label{volume}}

A \texttt{Volume} functional calculates the volume of volume elements.

\begin{lstlisting}
var lv = Volume()
\end{lstlisting}

\hypertarget{scalarpotential}{%
\section{ScalarPotential}\label{scalarpotential}}

The \texttt{ScalarPotential} functional is applied to point elements.

\begin{lstlisting}
var ls = ScalarPotential(potential, gradient)
\end{lstlisting}

You must supply two functions (which may be anonymous) that return the
potential and gradient respectively.

This functional is often used to constrain the mesh to the level set of
a function. For example, to confine a set of points to a sphere:

\begin{lstlisting}
import optimize
fn sphere(x,y,z) { return x^2+y^2+z^2-1 }
fn grad(x,y,z) { return Matrix([2*x, 2*y, 2*z]) }
var lsph = ScalarPotential(sphere, grad)
problem.addlocalconstraint(lsph)
\end{lstlisting}

See the thomson example for use of this technique.

\hypertarget{linearelasticity}{%
\section{LinearElasticity}\label{linearelasticity}}

The \texttt{LinearElasticity} functional measures the linear elastic
energy away from a reference state.

You must initialize with a reference mesh:

\begin{lstlisting}
var le = LinearElasticity(mref)
\end{lstlisting}

Manually set the poisson's ratio and grade to operate on:

\begin{lstlisting}
le.poissonratio = 0.2
le.grade = 2
\end{lstlisting}

\hypertarget{equielement}{%
\section{EquiElement}\label{equielement}}

The \texttt{EquiElement} functional measures the discrepency between the
size of elements adjacent to each vertex. It can be used to equalize
elements for regularization purposes.

\hypertarget{linecurvaturesq}{%
\section{LineCurvatureSq}\label{linecurvaturesq}}

The \texttt{LineCurvatureSq} functional measures the integrated
curvature squared of a sequence of line elements.

\hypertarget{linetorsionsq}{%
\section{LineTorsionSq}\label{linetorsionsq}}

The \texttt{LineTorsionSq} functional measures the integrated torsion
squared of a sequence of line elements.

\hypertarget{meancurvaturesq}{%
\section{MeanCurvatureSq}\label{meancurvaturesq}}

The \texttt{MeanCurvatureSq} functional computes the integrated mean
curvature over a surface.

\hypertarget{gausscurvature}{%
\section{GaussCurvature}\label{gausscurvature}}

The \texttt{GaussCurvature} computes the integrated gaussian curvature
over a surface.

\hypertarget{gradsq}{%
\section{GradSq}\label{gradsq}}

The \texttt{GradSq} functional measures the integral of the gradient
squared of a field. The field can be a scalar, vector or matrix
function.

Initialize with the required field:

\begin{lstlisting}
var le=GradSq(phi)
\end{lstlisting}

\hypertarget{nematic}{%
\section{Nematic}\label{nematic}}

The \texttt{Nematic} functional measures the elastic energy of a nematic
liquid crystal.

\begin{lstlisting}
var lf=Nematic(nn)
\end{lstlisting}

There are a number of optional parameters that can be used to set the
splay, twist and bend constants:

\begin{lstlisting}
var lf=Nematic(nn, ksplay=1, ktwist=0.5, kbend=1.5, pitch=0.1)
\end{lstlisting}

These are stored as properties of the object and can be retrieved as
follows:

\begin{lstlisting}
print lf.ksplay
\end{lstlisting}

\hypertarget{nematicelectric}{%
\section{NematicElectric}\label{nematicelectric}}

The \texttt{NematicElectric} functional measures the integral of a
nematic and electric coupling term integral((n.E)\^{}2) where the
electric field E may be computed from a scalar potential or supplied as
a vector.

Initialize with a director field \texttt{nn} and a scalar potential
\texttt{phi}: var lne = NematicElectric(nn, phi)

\hypertarget{normsq}{%
\section{NormSq}\label{normsq}}

The \texttt{NormSq} functional measures the elementwise L2 norm squared
of a field.

\hypertarget{lineintegral}{%
\section{LineIntegral}\label{lineintegral}}

The \texttt{LineIntegral} functional computes the line integral of a
function. You supply an integrand function that takes a position matrix
as an argument.

To compute \texttt{integral(x\^{}2+y\^{}2)} over a line element:

\begin{lstlisting}
var la=LineIntegral(fn (x) x[0]^2+x[1]^2)
\end{lstlisting}

The function \texttt{tangent()} returns a unit vector tangent to the
current element:

\begin{lstlisting}
var la=LineIntegral(fn (x) x.inner(tangent()))
\end{lstlisting}

You can also integrate functions that involve fields:

\begin{lstlisting}
var la=LineIntegral(fn (x, n) n.inner(tangent()), n)
\end{lstlisting}

where \texttt{n} is a vector field. The local interpolated value of this
field is passed to your integrand function. More than one field can be
used; they are passed as arguments to the integrand function in the
order you supply them to \texttt{LineIntegrand}.

\hypertarget{areaintegral}{%
\section{AreaIntegral}\label{areaintegral}}

The \texttt{AreaIntegral} functional computes the area integral of a
function. You supply an integrand function that takes a position matrix
as an argument.

To compute integral(x*y) over an area element:

\begin{lstlisting}
var la=AreaIntegral(fn (x) x[0]*x[1])
\end{lstlisting}

You can also integrate functions that involve fields:

\begin{lstlisting}
var la=AreaIntegral(fn (x, phi) phi^2, phi)
\end{lstlisting}

More than one field can be used; they are passed as arguments to the
integrand function in the order you supply them to
\texttt{AreaIntegrand}.

\hypertarget{floryhuggins}{%
\section{FloryHuggins}\label{floryhuggins}}

The \texttt{FloryHuggins} functional computes the Flory-Huggins mixing
energy over an element:

\begin{lstlisting}
a*phi*log(phi) + b*(1-phi)+log(1-phi) + c*phi*(1-phi)
\end{lstlisting}

where a, b and c are parameters you can supply. The value of phi is
calculated from a reference mesh that you provide on initializing the
Functional:

\begin{lstlisting}
var lfh = FloryHuggins(mref)
\end{lstlisting}

Manually set the coefficients and grade to operate on:

\begin{lstlisting}
lfh.a = 1; lfh.b = 1; lfh.c = 1;
lfh.grade = 2
\end{lstlisting}
